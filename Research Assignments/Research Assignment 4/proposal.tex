%% Beginning of file 'sample631.tex'
%%
%% using aastex version 6.3
\documentclass[twocolumn]{aastex631}

\newcommand{\vdag}{(v)^\dagger}
\newcommand\aastex{AAS\TeX}
\newcommand\latex{La\TeX}

\begin{document}

\title{Fate of Solar System-Like Stars in MW-M31 Merger}

\author[0000-0003-4242-8606]{Ansh R. Gupta}
\affiliation{Steward Observatory, University of Arizona, 933 N Cherry Avenue, Tucson, AZ 85721, USA}
\date{April 6, 2023}

\keywords{Local Group --- Merger remnant --- Stellar disk --- Stellar bulge --- Hernquist profile}

\section{Introduction} \label{sec:intro}

The Local Group (LG) is a group of galaxies dominated in mass by the Milky Way (MW) and Andromeda galaxy (M31). The LG's third most massive object, the Triangulum galaxy (M33), is a satellite of M31 \citep{2007gitu.book.....S}. Current estimates indicate that the MW and M31 will merge in approximately 5.86 Gyr; many stars will remain within the merger remnant, which consists of the bulk of the stars, gas, and dark matter that coalesce into a single galaxy following the merging event \citep{2012ApJ...753....9V}. The expected morphology of the MW and M31 during and after the merger has far-reaching consequences. Components of spiral galaxies include the stellar disk, a relatively thin region of space along a galactic plane including significant amounts of dust, gas, and stars; the stellar bulge, a more spherical, compact region at the center of spiral galaxies; and a halo, which contains a large amount of dark matter along with gas and stars \citep{2018MNRAS.478..611B}. Studying the evolution of these galaxy components can yield insight into how interacting galaxies grow and change over time \citep{2012ApJ...746..108T}, give information about the migration of stars throughout galaxies over time \citep{2012MNRAS.426.2089R}, and produce statistical predictions about the long-term trajectories of the Sun and similar stars \citep{2012ApJ...753....9V}. Here, I propose methods to visualize the eventual fate of a specific subset of bodies within M31. Specifically, I aim to determine whether stars at a similar distance from the M31 galaxy core as the Sun's galactocentric distance remain within the merger remnant, are ejected, or are captured by M33. To aid in the visualization of these results, movies of the merger event highlighting the selected particles will be created. These movies will be made displaying a simulated view centered on a selection of particles that are ejected, remain in the system, or are captured by M33.

Selecting specific stellar populations and determining their outcomes following a merger is a powerful method to probe how the composition and morphology of interacting galaxies change over time \citep{2012ApJ...753....9V}. While I aim to select only Sun-like stars in this work, these methods can be modified to probe the general kinematics of stars stars in a galactic disk, bulge, or halo. The varying interactions of these galaxy components may help predict the long term evolution and growth of merging galaxies \citep{1996ApJ...462..576D}. Visualizations can help display patterns that might be difficult to notice in data, contributing to the maturity and success of a variety of science projects \citep{2017PASP..129e8002P}. They also serve as a vital tool for public outreach, promoting education, scientific literacy, and public engagement with the research process \citep{2017PASP..129e8007R}. Methods to understand the long term evolution of galaxies by component are necessary to make astrophysical predictions about the past and future \citep{2012ApJ...753....7S}. Specifically, galaxy evolution is the set of processes through which galaxies formed, generating specific structures, and which cause changes in kinematics and morphology over time \citep{2007gitu.book.....S}. Indeed, the definition of a galaxy is a gravitationally bound collection of stars, gas, dust, and dark matter \citep{2012AJ....144...76W}. Investigating the mechanisms which cause these constituents to form structures and evolve is a major area of study in astronomy. Knowledge of the evolution of Sun-like stars in multiple galaxies will contribute to this field of interest. 

Current state-of-the-art simulations suggest that the majority of Sun-like stars in the Milky Way will remain within the merger remnant. However, analysis reveals that these objects will be distributed much more evenly throughout the remnant. \citet{2012ApJ...753....9V} use a canonical N-body simulation of the merger to show that a large majority of objects in a Sun-like orbit within the MW $(>85\%)$ migrate outwards following the merger, but most remain within $\sim 50$ kpc. A minority $(<0.1\%)$ are ejected out to a large radius $(>100$ kpc), but all candidates in the simulation remained gravitationally bound to the system. Although approximately $20\%$ of the candidates in the simulation pass through M33, none are accreted by the galaxy. Many of the selected stars are dislodged into "stellar streams", long tails of stars cast out during the merging process. Several stellar streams caused by the tidal disruption of dwarf galaxies have been discovered in the MW halo \citep[and references therein]{2006ApJ...642L.137B}. \citet{2014ApJ...787...19M} show that some streams caused by the tidal disruption of dwarf galaxies show a strong radial distance gradient, which is consistent with the low proportion of highly distant Sun-like stars from the merger simulation. Snapshots of the simulation from \citet{2012ApJ...753....9V} are shown in Figure \ref{fig1}, which visually summarize the possible fates of Sun-like stars and display the structure of the stellar streams.

\begin{figure}[ht!]
\plotone{figure1.jpg}
\caption{Individual frames from an N-body simulation of the MW-M31 merger, projected onto the galactocentric plane with the origin at the MW center of mass. The time of each snapshot is shown in the top left, with dotted lines indicating galaxy orbital paths. Sun-like stars are shown in red. Most of the selected stars migrate outwards, some are cast into stellar streams, and a small fraction are ejected far away. All objects remain gravitationally bound to the system. Figure adapted from \citet{2012ApJ...753....9V}.
\label{fig1}}
\end{figure}

Although previous work investigates the future of MW Sun-like stars, the fate of such objects in the M31 disk has not yet been studied. Furthermore, it's not clear whether analogous systems in M33 are captured in tidal tails, accreted by the merger remnant, remain part of M33, or are otherwise ejected. Furthermore, previous simulations have not included other LG galaxies, such as the Large and Small Magellanic Clouds \citep{1996ApJ...462..576D}\citep{2008MNRAS.386..461C}\citep{2012ApJ...753....9V}. An open question remains as to whether these other galaxies may have an affect on the merger, whether Sun-like stars will pass through them, and whether they may be disrupted by or impact the MW before the merger takes place. As the proper motions of other LG satellite galaxies have been better constrained, including them in future N-body simulations is a topic of interest \citep{2015IAUS..311....1V}.

\section{This Project \label{sec:this-project}}
In this work, I will select sun-like particles in the disk of M31 and M33 and track their motions over the course of the merging event using N-body simulation data supplied from \citet{2012ApJ...753....9V}. Moreover, I will create visualizations to demonstrate the bulk migration of these particles over time. These particles are selected to have a similar distance from the M31 galactic center as the Sun's distance to the center of the MW ($\sim 8.29$ kpc) and a circular velocity as determined in the Methodology section \citep{2012ApJ...753....8V}.

This study will address the question of where a specific subset of stars will reside after the MW-M31 collision. Furthermore, this study will demonstrate if similar stars in M33 are accreted by the remnant, are ejected out of the LG, or are cast into tidal tails.

Knowledge of the future kinematics of this group of objects will help determine the eventual fate of Sun-like particles in the disk of M31 following the merger and contribute to the understanding of how specific galactic disk components evolve over time following major merger events. These visualizations can be used to qualitatively examine the morphological distribution of the selected objects following the merger, which can visually display the possible and most likely far-future orbits of Sun-like stars in M31 and M33. 

% Quantitative analysis using the particle positions can be conducted to solidify these results, if necessary. In addition, movies of the night sky view from the perspective of a few sample particles following various trajectories will be created.

\section{Methodology} \label{sec:methodology}
Simulation data is supplied from \citet{2012ApJ...753....9V}, in which an N-body simulation is used to determine the outcome and properties of the MW-M31 merger in the next 10 Gyr from the current time. An N-body simulation is one in which the positions of a number of particles over time are directly calculated using numerical integration considering the mutual forces between each each pair of particles. Each particle represents a specified mass of stars or dark matter. Gas and dust, which contribute only slightly to the overall mass of the galaxy, are neglected. \citet{2012ApJ...753....9V} use canonical initial conditions, those which are approximately in the middle of the range constrained by previous works. Furthermore, a Monte Carlo approach is used to quantify the effects of varying these parameters within these allowed ranges.

Sun-like particles will be selected by applying a series of filters to the total sample of stars from the simulation. First, the selection will be limited to particles with a similar radius to the Sun's galactocentric distance, within a set tolerance. Then, the circular velocities and out-of-plane velocities of each particle will be computed and restricted to lie within specified ranges given by specific assumptions, as described in more detail below. The remaining particle selection will be the Sun-like particles used for analysis. Plots will be made showing the positions of this group of objects within M31 and M33 at each snapshot of the N-body simulation, and the plots will be stitched together into a movie to complete the final visualization. An example frame is shown in Figure \ref{fig2}.

\begin{figure}[ht!]
\plotone{figure2.png}
\caption{Example frame produced using Matplotlib showing a possible view of the visualization, with disk particles of the MW, M31 and M33 shown. The proposed visualization will highlight Sun-like particles of M31 in a different color. This style of visualization will work for the 3-D night-sky views from Sun-like M31 particles, which are intended to display the perspective of those objects. A Mollweide projection will be used to generate 2-D night sky views. Simulation data taken from \citet{2012ApJ...753....9V}.
\label{fig2}}
\end{figure}

To select Sun-like particles in M31 and M33 and produce visualizations tracking their orbital evolution, a few specific codes must be created. An existing code uses iterative spheres of shrinking volume to calculate the center of mass (COM) of each galaxy to avoid contamination from ejected stars. Using these COM coordinates, the distance of each galaxy particle from its corresponding center can be determined. A new code will be created to determine the circular speed of particles by determining the tangential component of their overall velocity. This can be accomplished by using a previously developed method to rotate the plane of a given galaxy such that its total angular momentum vector points along a specified axis. Then, the perpendicular components of the velocity can be used to give the circular speed. Specifically, if the disk is rotated such that the angular momentum vector lies along the z-axis, the circular speed is determined as follows.

\begin{equation}
    v_{circ} = \sqrt{{v_x}^2 + {v_y}^2}
\end{equation}

Using the determined radii and circular speeds, another code will select particles that match the Sun's distance from the galactic center within a set tolerance. This tolerance will initially be fixed at $\sim 10\%$, but can be adjusted to broaden or narrow the selection. Then, using an additional previously developed code, the expected circular velocity at the selected radius values will be computed by assuming a Hernquist potential for each galaxy \citep{1990ApJ...356..359H}. Circular speeds of candidate particles will be compared to this figure within a selected tolerance. At this time, the time-evolution of the Sun's orbit through the MW disk in the z-axis direction is not well constrained; instead, candidate Sun-like particles will be filtered by limiting their z-axis velocities to $30 km s^{-1}$ to exclude objects moving with significant out-of-plane velocity, following \citet{2012ApJ...753....9V}. Using the Python software Matplotlib, all particles will be plotted, with the Sun-like stars being assigned a separate color. The visualization will proceed centered on the MW COM. These frames can then be stitched together into a movie using Adobe Premiere. Plots including axis labels and units will be generated as well. Night sky views from the perspective of a given particle can be constructed by re-centering the simulation. Transformed coordinates for each star are the relative positions of each to the particle of interest. To produce a night-sky view, a 2-dimensional plot can be made using a Mollweide projection, which maps the 3-D sphere containing all particles onto a plane, and a 3-D movie can be made in software such as Adobe After Effects using a list of the relative positions of each particle for each frame. 

The visualizations produced will directly show where Sun-like stars in M31 and M33 are located over time. Qualitatively, these movies will show how these stars migrate radially throughout the disk before the merger. Additionally, they will help understand where this subset of objects ends up in the LG, whether as part of the merger remnant, a component of tidal tails, or among the particles ejected out of the system. Moreover, these visualizations will add to the existing visual aids directly demonstrating the future fate of the MW, M31, and M33.

M31 is of comparable size to the MW, so one may expect that the trajectories of Sun-like stars are similar for both galaxies. One significant difference is that the M31 spin axis is nearly perpendicular to the orbital plane of the MW and M31, which also roughly defines the plane of the collision \citep{2012ApJ...753....9V}. This is in contrast to the MW, which has a closer orbital alignment. These differences may cause a change in the dynamics of M31 Sun-like stars, as stars off-axis from the collision may be more significantly disrupted. Also, as M33 is a satellite galaxy of M31, ejected stars may have a greater probability of being captured by it. However, overall, it seems reasonable to assume that the outcomes of these stars will be similar to those in the MW due to the similarities of the two galaxies.

\begin{acknowledgments}
Thank you to Professor Gurtina Besla and TA Hayden Foote for instructing the Spring 2023 semester ASTR 400B course at the University of Arizona.
\end{acknowledgments}

\bibliography{bibliography}{}
\bibliographystyle{aasjournal}

\end{document}

% End of file `sample631.tex'.
