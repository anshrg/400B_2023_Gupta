\documentclass[12pt]{article}
\usepackage[margin=1in]{geometry}
\usepackage{adjustbox}
\usepackage[center]{titlesec}

\begin{document}
\thispagestyle{empty}
\section*{Homework 3}

\noindent \textit{1. How does the total mass of the MW and M31 compare in this simulation? What galaxy component dominates this total mass?}\\

\noindent The total masses of the Milky Way and M31 are approximately the same. Both masses are dominated by the dark matter halo component.\\
\newline

\noindent \textit{2. How does the stellar mass of the MW and M31 compare? Which galaxy do you expect to be more luminous?}\\

\noindent The Milky Way has a lower disk and bulge mass than M31, so has less total stellar mass. From this, you would expect M31 to be more luminous.\\
\newline

\noindent \textit{3. How does the total dark matter mass of MW and M31 compare in this simulation (ratio)? Is this surprising, given their difference in stellar mass?}\\

\noindent The Milky Way has more dark matter than M31, both absolutely and as a ratio. To me, this is somewhat naively surprising, as I would expect a galaxy with more dark matter to have more stellar mass as well.\\
\newline

\noindent \textit{4. What is the ratio of stellar mass to total mass for each galaxy (i.e. the Baryon fraction)? In the Universe, ${\Omega}_b$/${\Omega}_m$ $\sim16\%$ of all mass is locked up in baryons (gas \& stars) vs. dark matter. How does this ratio compare to the baryon fraction you computed for each galaxy? Given that the total gas mass in the disks of these galaxies is negligible compared to the stellar mass, any ideas for why the universal baryon fraction might differ from that in these galaxies?}\\

\noindent The Baryon fraction of the Milky Way is about $4.1\%$. For M31 the fraction is approximately $6.7\%$, and for M33 it's $4.6\%$. These percentages are all significantly lower than the fraction for the overall Universe. One idea for why the overall fraction of baryonic matter is higher in the overall Universe is that the galaxy masses don't include the intergalactic medium, which is poor in dark matter.\\

\begin{table}[h]
    \centering
    \caption{Table of mass components for Local Group galaxies}
    \begin{adjustbox}{width=\textwidth}
        \begin{tabular}{|c|c|c|c|c|c|}
        \hline
        Galaxy Name & Halo Mass ($10^{12} M_{\odot}$) & Disk Mass ($10^{12} M_{\odot}$) & Bulge Mass ($10^{12} M_{\odot}$) & Total Mass ($10^{12} M_{\odot}$) & $f_{bar}$\\ \hline
        Milky Way & 1.975 & 0.075 & 0.010 & 2.060 & 0.041\\
        M31 & 1.921 & 0.120 & 0.019 & 2.060 & 0.067\\
        M33 & 0.187 & 0.009 & 0.000 & 0.196 & 0.046\\
        Local Group & 4.083 & 0.204 & 0.029 & 4.316 & 0.054\\ \hline
        \end{tabular}
    \end{adjustbox}
\end{table}

\end{document}
